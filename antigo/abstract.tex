
\begin{abstract}
      Nowadays the meta heuristcs have been used for its simply of implementation
      and for be able to be applied in a very large range of problems whith the most
      levels of complexity. In this work, we have study the technique named Particle
      Swarm Optimization (PSO). Which one, as occurs in the most of meta heuristcs, the
      quality of application of this technique have to define some parameters that have
      direct influence in the performance of algorithm, like the inertia weight that
      prevent that the particle change its direction instatly, and the swarm learning
      ability and its own. Those parameters was evaluted by a group of tests largely used
      on literature and whith the goal of observe how the algorithm behave in each group
      of parameters in diferents situations applied.

      Furthermore, a larger study about the technique has been done evaluting algorithms
      that adapt their parameters during the execution, thus searching a tool more effective
      applied on diferents groups of problems. Algorithms that adapt the inertia weight,
      for instance, generally allow that the particle keep more free in the opening
      execution e became more restrective during the search.
% 	Al\'em disso, um estudo mais amplo a respeito da t\'ecnica foi feito avaliando 
% 	algoritmos que adaptam seus par\^ametros no decorrer da execu\c c\~ao, buscando 
% 	assim uma ferramenta mais eficaz aplicada em diferentes classes de problemas. 
% 	Algoritmos que adaptam a in\'ercia por expemplo, geralmente permitem que a 
% 	part\'icula permanessa mais livre no in\'icio da execu\c c\~ao do programa e tornam-se 
% 	mais restritivas com o decorrer da busca.

	Particles on its turn, do not depend olny of the inertia wight to evolve. Its learning
	is give by two parameters known as cognitive aceleration and social aceleration, which one,
	define how much the particle is influenced by the swarm and how much she must search the 
	best by	itself. (Thus, if implement more social, in other words, that give more importance
	to the swarm learn, it can lead to failure of all group in case it is lost.) So, it is
	clear that the definition of those parameters commits the final result of the technique.
% 	
% 	As part\'iculas por sua vez, n\~ao dependem aprenas da in\'ercia para evoluirem. Seu
% 	aprendizado se da por dois par\^ametros conhecidos como acelera\c c\~ao cognitiva e 
% 	aclera\c c\~ao social, que dizem o quanto uma part\'icula \'e influenciada para seguir o 
% 	aprendizado do enxame e o quanto ela deve buscar por si pr\'opria o ponto \'otimo.
% 	(Sendo assim, se uma execu\c c\~ao mais social, ou seja, que d\^e mais importancia para
% 	o aprendizado do enxame, pode levar ao fracasso de todo o grupo caso ele se perca.)
% 	Com isso, fica claro que a defini\c c\~ao desses par\^ametros compromete o resultado final da t\'ecnica.

	Beyond those types of adaptation, still exist algorithms that updates in diferents ways
	the particle's velocity, according with what those particles have learn, as well as
	inclusion of new necessary parameters to control the adaptation.
% 	
% 	Al\'em desses tipos de adapta\c c\~ao, existem algoritimos que atualizam a 
% 	velocidade das part\'iculas de maneiras diferentes de acordo com o que as 
% 	part\'iculas aprenderam, bem como a inclus\~ao de novos par\^ametros necess\'arios 
% 	ao controle da adapt\c c\~ao.

	There fore, thiw work aims to evalute various proposals around the definition of PSO
	parameters in a group of optimization problems without restriction, aiming identify proposals
	more adequate for diferents classes of problems.
% 	Portanto, este trabalho visa avaliar diversas propostas em torno da defini\c c\~ao 
% 	dos par\^ametros do PSO em um conjunto de problemas de otimiza\c c\~ao sem restri\c c\~oes, 
% 	visando identificar as propostas mais adequadas para diferentes classes de problemas.
\end{abstract}