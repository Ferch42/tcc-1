\chapter{Conclusão\label{cap:conclusao}}
    Tendo como foco principal a criação do \emph{Framework Lothus\{PHP\}} de acordo com a necessidade de criar uma forma ágil para desenvolvimento de sites e sistemas, o projeto cumpriu com os objetivos esperados, visto que, foram comprovadas as funcionalidades especificadas e validados os objetivos específicos.

    O objetivo geral do trabalho foi desenvolver um \emph{Framework} utilizando \emph{PHP} como principal linguagem de programação e o \emph{MVC} como padrão de projetos, e cujo o foco é facilitar a criação de sites e sistema, fornecendo bibliotecas capazes de cuidar de toda a parte genéria de um projeto, direcionando o foco do desenvolvedor apenas ao que for específico.
    Para tornar a proposta possível foram implementadas todas as tecnologinas aqui descritas, viabilizando de maneira eficiente a construção e implantação do \emph{Framework}, atendendo ao que foi proposto.

    O \emph{Framework} atendeu as expectativas esperadas, uma vez que, sistemas e sites foram implementados de maneira eficaz, realizando o que a ferramenta se propõe a fazer.

    Como qualquer projeto, algumas dificuldades foram encontradas para a realização deste trabalho. Entre tantas dificuldades podemos destacar a filtragens de tecnologias e ferramentas que foram utilizadas no \emph{Framework} com a finalidade de atender ao que foi proposto.

    Apesar das expectativas propostas e atingidas, das dificuldades encontradas e superadas, entendo que a conclusão deste trabalho não deve ficar restrito apenas para uso acadêmico, tendo em vista que, o mesmo já está sendo utilizado em diversos projetos ligado a empresas e desenvolvedores.

