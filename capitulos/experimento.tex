\chapter{Experimento com Framework existente\label{cap:experimento}}

    \section{Cake PHP\label{sec:cake-php}}
        \subsection{Descricao da ferramenta\label{sub:descricao-cake}}

            O CakePHP é um projeto de código aberto mantido por uma comunidade bastante ativa de desenvolvedores PHP. Possui uma estrutura extensível para desenvolvimento, manutenção e implantação de aplicativos. Utiliza o padrão de projeto MVC\emph{(Model-View-Controller)} e ORM\emph{(Object-relational mappring)} com os paradigmas das convenções sobre configurações.


        \subsection{Objetivo\label{sub:objetivo-cake}}

            CakePHP tem como objetivo principal a simplificação do processo de desenvolvimento e construção de aplicações web, utilizando um núcleo onde organiza o banco de dados e alguns recursos que reduzem a codificação pelo desenvolvedor. Alguns desses recursos são a validação embutida, ACLs \emph{(lista de controle de acesso)}, segurança, manipulação de sessão e cache de Views e sanitização de dados.


        \subsection{Características\label{sub:caracteristicas-cake}}
            \begin{itemize}
                \item Possui licença flexível ... completar
                \item Ativo e com comunidade amigável
                \item Compatível com PHP5
                \item Geração de CRUD (\emph{Create, Read, Update and Delete, ou Criar, Ler, Atualizar e Excluir})
                \item Funciona em qualquer subdiretório web, com poucas configurações no apache
                \item Utiliza templates
            \end{itemize}