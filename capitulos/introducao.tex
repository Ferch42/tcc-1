\chapter{Introdução\label{cap:introducao}}
    Devido à grande necessidade de entregar projetos de grande porte e com prazos consideravelmente baixos, foi percebida a necessidade de se encontrar soluções que facilitassem esse desenvolvimento.

    A primeira atitude a ser tomada foi a criação de um arquivo que reunia diversas funcionalidades, afim de facilitar futuros projetos, onde processos que se repetiam diversas vezes eram colocados em funções que poderiam ser usadas em novos projetos.

    Aplicações em geral precisam de um padrão mais significativo como forma estrutural de um projeto, deixando claro que a criação de um arquivo contendo todas as funções do projeto não era o melhor padrão a ser seguido. Este trabalho apresenta, como uma de suas justificativas, uma pesquisa profunda que reune novos padrões para os processos e \emph{Frameworks} web que poderiam ser mais úteis para um desenvolvimento ágil.

    No final dessas pesquisas iniciais, alguns \emph{Frameworks} foram testados e o CakePHP passou a ser usado como padrão. O CakePHP utiliza o padrão de projeto MVC \emph{(Model, View, Controller)} que é um modelo de arquitetura de software que tem como objetivo básico separar a lógica de negócio da aplicação.

    Os \emph{Frameworks} são sempre muito robustos e com diversos tipos de funcionalidades, e com o CakePHP não é diferente. Com uma vasta documentação e uma quantidade considerável de arquivos em seu projeto mais simples, este passou a ser um problema ao invés de solução quando se busca um total domínio em uma aplicação.

    Percebeu-se então a real necessidade de criar um Framework onde se tenha total controle de todas as funcionalidades, mantendo o padrão MVC, porém criando as próprias funcionalidades, mesmo que baseado em funcionalidades de outros \emph{Frameworks}.

    Identificar o problema foi o primeiro e principal passo para se iniciar o desenvolvimento do Framework, que se encontra sempre em evolução com novas implementações que resolvam determinados problemas.
