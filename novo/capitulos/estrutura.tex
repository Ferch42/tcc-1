\chapter{Estrutura e funcionamento do Framework\label{cap:detalhamento-projeto}}

    \section{Processo de instalação\label{sec:processo-instalacao}}
        O \emph{Framework Lothus\{PHP\}} permite ao desenvolvedor a possibilidade de escolha entrei dois níves para a aplicação. O primeiro nível permite a instalação do \emph{Framework} da forma mais simples, instalando utilitário que foca em um desenvolvimento direcionado ao backend do projeto, integrando facilidade com a troca de informações com o banco de dados, desenvolvimento através do MVC, URLs amigáves entre outras funcionalidades descritas ao longo deste trabalho.

        Essa instalação é feita através do repositório remoto chamado \emph{Github}, que se encontra no seguinte endereço online:

        \emph{https://github.com/guilouro/FRAMEWORK-PHP}




    \section{Estrutura de pastas\label{sec:estrutura-pastas}}

    \section{Divisão Backend - frontend\label{sec:back-front}}

        \subsection{Comandos do Grunt\label{sub:comandos-grunt}}

    \section{System\label{sec:estrutura-pastas}}

    \section{Model\label{sec:estrutura-pastas}}

    \section{View\label{sec:estrutura-pastas}}

    \section{View\label{sec:estrutura-pastas}}

    \section{Template\label{sec:estrutura-pastas}}
